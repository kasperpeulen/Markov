\exercise{1.1.3iiia}{Suppose now that $Z_0,Z_1,...$ are independent, identically distributed random variables such that $Z_i=1$ with probability $p$ and $Z_i=0$ with probability $1-p$. Set $S_0=0, S_n=Z_1,...,Z_n$. In each of the following cases determine whether $(X_n)_{n \geq 0}$ is a Markov chain. In the cases where $(X_n)_{n \geq 0}$ is a Markov chain find its state-space and transition matrix, and in the cases where it is not a Markov chain give an example where $P(X_{n+1}=i |X_n=j, X_{n-1} =k)$ is not independent of $k$. First one $X_n=Z_n$.}

\beginproof
The state space is $\{0,1\}$. As we now that $Z_0,Z_1,...$ are independent, it is clear that $Z_n$ doesn't depend on $Z_{n-1}$. It's transition matrix is:

\[\begin{pmatrix}
1-p & p\\ 
1-p & p
\end{pmatrix}\]
Which is clear from the definition.\endproof

\exercise{1.1.3iiib}{Let $X_n=S_n = Z_1 + ... + Z_n$}

\beginproof
The state space is $\{0,1,...\}$.

\begin{align*}
P(S_{n+1} = i_{n+1} | S_0 = i_0,...,S_{n}=i_{n}) &= P(S_{n}+Z_{n+1}=i_{n+1} | S_n=i_n )   
\end{align*}\endproof